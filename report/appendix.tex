\chapter{Appendix}

\section{Bridge Design Pattern}
The Bridge design pattern allows us to separate the abstraction from the implementation, thus
allowing us to hide the implementation from the client and only providing the abstraction~\citep{oodesign}.
<<<<<<< HEAD
=======

>>>>>>> 0e226c37b42fb6f41ac7d05353a8cc6b61a042b2
Use when:
\begin{itemize}
	\item we want run-time binding of the implementation
	\item we have a collection of related classes resulting from a coupled interface and numerous
	implementations
	\item we want to share an implementation among multiple objects
	\item we need to map orthogonal class hierarchies
\end{itemize}

\section{State Design Pattern}
The state pattern allows an object to change it's behaviour when it changes it's internal state. The main object is called the context and this is the object that keeps the state. When the state needs to be changed the state currently being ran can update that state of the context.
