\chapter{Added Value}

  \section{Git \& Github}
  We used Git for handling version control with ease and Github to share the project with each other.
  We also submitted pull requests which were subjected to code review. This made us all aware of
  changes and allowed us to discuss issues in a relevant place.

  \section{LaTex}
  We used LaTex for creating the report. LaTex is much more version control friendly than other
  software such as Microsoft Word and it also runs on all platforms including Linux with not
  discrepancies in formatting. It also allowed us to reference external diagram exports so that
  the document never had a stale version.

  \section{Dependency Management}
  We used \emph{pip}, a package manager for Python, which allowed us to declare our dependencies.
  This meant that all team members could easily install new dependencies.

  \section{Automated Unit Testing}
  We wrote unit tests for many of the core modules we wrote. We used the Python inbuilt unit testing
  framework as well as \emph{nose2}, an automated framework which wraps the inbuilt one with extra
  functionality. This allowed us to verify that code changes were non-breaking.

  \section{Flask Framework}
  Flask is a microframework for web development in Python. We used it to implement the REST API
  which controls the pipeline.

  \section{NoSQL Database}
  We decided that SQL was not appropriate for our scenario so we decided to use DynamoDb which is a
  NoSQL database.

  \section{Deployment Configuration}
  The project can be deployed with different configurations for different environments such as
  testing, development and production. This allows us to set a different log level and a different
  database endpoint.

  \section{Pluggable Adapter}
  We implemented the pluggable Adapter to create an abstraction layer between our application and
  the chosen NoSQL database. This means that any database which supports the set of CRUD operations
  defined in our interface can be used by just implementing the required hooks in the pluggable
  adapter.

  \section{Reflection \& Metaprogramming}
  \label{added_value_reflection}
  % ref impl.
  We used reflection to allow us to create a flexible ORM-like interface for clients.
  Clients can implement their own interceptors, but in the context of our scenario these need
  configuration. Since the configuration is quite unstructured it is difficult to model.
  We implemented a class called \emph{Config Model} which served as a base class for all other configs.
  This class abstracts away all the complications associated with storing, retrieving and restoring
  unstructured objects. Clients use decorators to \emph{tag} getters for values that should be stored
  in the database as well as any other meta data associated with the field.
  It is extremely flexible and even allows nested data structures using the composite design
  pattern. A memento can be created and restored for any of the sub classes.

  Python has support for generics in type annotations, but there is no way to access this information
  at run time. We created a base class and used reflection to retrieve the information so that it
  could be used for something useful such as a generic factory.

  Python \emph{has} a limited way in supporting method overloading with the introduction of type
  annotations, but once again it is of little use beyond linting in a text editor.
  The current implementation requires stub methods with the correct types followed by a \emph{catch
  all} method which actually does instance of checks. This is a really ugly approach.
  We implemented a decorator which added the functionality that one would expect. The decorator
  uses reflection to find the parameter types and when a method is called it dispatches to the
  appropriate implementation. In other words, it abstracts out the instance-of checks.
  Whilst it cannot support complex generics due to type erasure, it works perfectly well
  for all other types and supports polymorphism which is sufficient to support most use cases
  and design patterns such as the Visitor.


\chapter{Problems Encountered}

  \section{Object Oriented Programming in Python}
  Python is a duck-typed language which has some drawbacks when trying to implement an OO-design.
  Recently a standard was released for Python which introduced \emph{type annotations}.
  We decided to try it for our project since it offered what seemed to be static types and support
  for generics.
  Unfortunately, most of the functionality has no benefit at runtime and is mostly ignored. It is
  provided to support a limited static typing for use with text editors that support it.

  This caused some problems when trying to implement certain design patterns or when using generics.
  However, we were able to overcome some of these limitations by introducing an overload
  decorator and implementing a generic base class which gave access to generic types at runtime
  using reflection. This is discussed in more detail in \ref{added_value_reflection}.
