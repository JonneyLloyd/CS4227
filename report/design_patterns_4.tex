\chapter{Design Patterns}
TODO context and consequences and support for scenario.
TODO generic factory, command
TODO code snippet of Generic Factory Method, Composite, Command, State, Interceptor

\section{Creational}
Creational design patterns are design patterns that deal with object creation
mechanisms, trying to create objects in a manner suitable to the situation. The basic form of object
creation could result in design problems or added complexity to the design. Creational design patterns
solve this problem by controlling this object creation~\citep{OODesign}.

\subsection{Abstract Factory}
The Abstract Factory design pattern adds another layer of abstraction to the Factory design pattern.
The Abstract Factory provides us with a framework that allows us to create families of related or
dependent objects without specifying their concrete classes. So at runtime, the abstract factory is
coupled with any desired factory which can create objects of desired type~\citep{OODesign}.

We used the Abstract Factory to create our server and load extensions.


\begin{lstlisting}[language=Python]
class AppFactory(object):

	@staticmethod
	def create_app(config: ServerConfig) -> Flask:
	app = Flask(__name__)
	app.config.from_object(config)
	with app.app_context():
		AppFactory._load_extensions_before(app)
		AppFactory._load_controllers(app)
		AppFactory._load_extensions_after(app)
	return app

	@staticmethod
	def _load_extensions_after(app: Flask) -> None:
		api.init_app(app)
		dynamo.init_app(app)
		dynamo.create_all()
\end{lstlisting}


\subsection{Generic Factory Method}
The generic factory method allows the usage of generics to parameterise a class so that it can later create instances
of that generic type, where the concrete type cannot be anticipated. We implemented this in our interceptors so
that they could be supplied with a memento and create an instance the appropriate configuration class to restore the
memento into. This allows our code to be easily extended by third parties without changes to the framework.

\begin{lstlisting}[language=Python]
	T = TypeVar('T', bound=ConfigModel)
	class ConfigurableInterceptor(GenericABC, Generic[T]):
	    """A generic base class responsible for creating the appropriate ConfigModel for interceptors.
	    """
	    @property
	    def config(self) -> ConfigModel:
	        return self._config

	    @config.setter
	    def config(self, config: ConfigMemento) -> None:
	        self._config = self._create_config(config)

	    def _create_config(self, config: ConfigMemento) -> T:
	        clazz: Type = self._get_generic_type(T)
	        c: T = clazz()
	        c.set_memento(config)
	        return c


U = TypeVar('U', bound=ConfigModel)
class SourceInterceptor(ConfigurableInterceptor[U], Generic[U]):
		"""Interface for clients - Associated with a particular dispatcher/interception point.
		"""

    @abstractmethod
    def on_source(self, context: SourceContext) -> None:
        ...


class ClientInterceptor(SourceInterceptor[ClientConfig]):

    def on_source(self, context: SourceContext) -> None:
				# Since the client parameterised it's parent generic class,
				# when a memento is restored we can access the our ClientConfig
				# without the framework knowing anything about it.
        print(f"The ClientInterceptor received an event! Here is ClientConfig '{self.config}'")


interceptor: ConfigurableInterceptor = ClientInterceptor()
# Reveals nothing about type info
memento: ConfigMemento = retreive_memento()
# Setter in base class will handle creation of the correct config
interceptor.config = memento

\end{lstlisting}

\section{Structural}
Structural design patterns are design patterns that ease the design by identifying a simple way to
realize relationships between entities~\citep{OODesign}.


\subsection{Composite}
The composite pattern is used to group objects into tree structures to represent whole-part hierarchies.
This pattern lets clients treat individual objects and compositions of objects uniformly ~\citep{sourcemaking}.
We used composition for modeling configuration. A core part of making our framework extensible and easy to use
was to provide an easy mechanism for providing configuration to modules. The difficulty is that the framework
cannot know the concrete implementation and configuration is typically very unstructured.

By implementing the composite design pattern in a base class in tandem with a combination of decorators and
reflection, we implemented an easy to use API for modeling client module configuaration with support for various
types and nested structures. This effectively allowed us to provide \textbf{ORM-like functionality} since this
structure works with our database and REST API without modifications in the framework.

\begin{lstlisting}[language=Python]
	class HostConfig(ConfigModel):

	    def __init__(self, ip: str=None, key: str=None) -> None:
	        self._ip = ip
	        self._key = key

	    @attribute_property('ssh_key')
	    def key(self) -> str:
	        return self._key

	    @key.setter
	    def key(self, key: str) -> None:
	        self._key = key

	    @attribute_property('ip_address')
	    def ip(self) -> str:
	        return self._ip

	    @ip.setter
	    def ip(self, ip: str) -> None:
	        self._ip = ip


	class FleetConfig(ConfigModel):

	    __documentname__ = 'fleet_of_hosts'

	    def __init__(self, region: str=None, hosts: List=None) -> None:
	        self._region = region
	        self._hosts = hosts if hosts else []

	    @attribute_property('hosts')
	    def hosts(self) -> List[HostConfig]:
	        return self._hosts

	    @hosts.setter
	    def hosts(self, hosts: List[DemoConfigChild]) -> None:
	        self._hosts = hosts

	    @attribute_property('region')
	    def region(self) -> str:
	        return self._region

	    @a.setter
	    def region(self, region: str) -> None:
	        self._region = region
\end{lstlisting}

The FleetConfig is composed of many HostConfigs. Any class that subclasses ConfigModel can be composed of
primative types and collections as well as other ConfigModel subclasses. The ConfigModel base class handles
all the required functionality for creating and restoring mementos for these classes so that the client does
not need to implement such methods. The contents of the memento can be easily stored in a no-SQL database.
We can also create a schema which can be provided to clients via the REST API. This also means, should we
decide to implement a GUI, that the GUI would not need to know anything about third party modules in order
to display their configuration and allow the user to modify it.

\subsection{Pluggable Adapter}
The pluggable adapter allows a system to be plugged in. It lets us incorporate a class into
existing systems that expected different interfaces.

We implemented the pluggable Adapter to create an abstraction layer between our application and
the chosen NoSQL database. This means that any database which supports the set of CRUD operations
defined in our interface can be used by just implementing the required hooks in the pluggable
adapter.

\begin{lstlisting}[language=Python]
class DocumentStore(ABC):

	@abstractmethod
	def _save(self, name: str, _list: List[dict]) -> None:
		...

	def save_pipeline(self, name: str, pipeline_memento: PipelineMemento) -> None:
		config_memento_list = self._extract_config_memento_list(pipeline_memento)
		self._save(name, config_memento_list)

class DynamoStore(DocumentStore):

	def __init__(self):
		self.store = dynamo

	def _save(self, name: str, _list: List[dict]) -> None:
		table = self.store.tables['mementos']
		table.put_item(Item={
			'type': name,
			'config': _list
		})
\end{lstlisting}

\section{Behavioural}
In software engineering, behavioural design patterns are design patterns that identify common communication patterns
between objects and realize these patterns. By doing so, these patterns increase flexibility in carrying out this
communication \citep{sourcemaking}.

\subsection{Command}
The command design pattern allows us to encapsulate an operation which can be passed as an object and executed at
a point where the code is oblivious to the operation being performed, other than knowing how to execute it.
In the context of our project we wanted the client to be able to provide a module (a config, interceptor pair)
which the framework could be responsible for creating. Since Python is a fully object-oriented language, where
primatives, functions, classes and even methods are actually objects, we implemented the command design pattern
in a similar way to delegates in C\#. Classes can be passes as arguments and later invoked with no arguments
(similar to an \emph{execute()} method) effectively calling a no-arg constructor. Implementing an actual class
for the command design pattern in Python is practically redundant and effectively just bloat given the syntax
available in the language. You may wonder what if the class requires parameters? This can easily be solved in
Python using the standard library and a function called \emph{partial} to pre-parameterise the class being passed
without actually creating it. This applies to functions/methods likewise. These conditions satisfy the intent of
the command.

\subsection{Memento}
The memento pattern allows us to capture and externalise an objects state so that the object can be returned to
this state later on.
Steps:
\begin{itemize}
	\item Ask originator to create and return a memento.
	\item Save the memento.
	\item Update the state of the originator.
	\item Retrieve a memento from storage.
	\item Ask originator to recreate its previous state using this memento.
\end{itemize}

We used the Memento design pattern to capture the pipeline state. These mementos were then stored in
a NoSQL database which allowed to us restore to such state an any stage.

\begin{lstlisting}[language=Python]
class Pipeline(PipelineBase):

	def set_memento(self, memento: PipelineMemento) -> None:
		self.config = memento.config

	def create_memento(self) -> PipelineMemento:
		memento = PipelineMemento()
		memento.config = self.config
		return memento

class PipelineMemento:

	@property
	def config(self) -> List[ConfigMemento]:
		return self._config

	@config.setter
	def config(self, value: List[ConfigMemento]) -> None:
		self._config = value

class ConfigMemento():

	@property
	def config(self) -> Optional[Dict[str, Any]]:
		return self._config or None

	@config.setter
	def config(self, value: Dict[str, Any]) -> None:
		self._config = value
\end{lstlisting}

Normally both a narrow and wide interface should be provided, but for two reasons this was inappropriate for us.
Firstly, Python is a duck-typed language. There is no way to enforce usage of interfaces. Although we have used
type annotations, which would make it look like static typing is present, methods from subclasses can still be
accessed i.e. we cannot cast to superclass types. Interfaces are also not considered idiomatic in the language.
Since interfaces cannot be used to enforce types, they are redundant and have no benefit.
Abstract classes are usually preferred if they add any functionality.
Secondly, since we implemented a client server architecture, we were required to store mementos in a database which
would require what is typically considered a wide interface. Again, since type information cannot be lost by
using interfaces and therefore casting is not possible (because it would be redundant) it is appropriate.
We considered serialization of the mementos but that would be worse given that serialised objects should never
appear in a database as they are language and implementation specific.

\subsection{Strategy}
Define a family of algorithms, encapsulate each one, and make them interchangeable. Strategy lets the
algorithm vary independently from clients that use it~\citep{OODesign}

We used the Strategy design pattern to add encryption to text.


\begin{lstlisting}[language=Python]
class Encryptor(ABC):

	@abstractmethod
	def encrypt(self, text: str) -> bytes:
		...

class FernetEncryptor(Encryptor):

	def __init__(self) -> None:
		self._key = Fernet.generate_key()
		self._suite = Fernet(self._key)

	def encrypt(self, text: str) -> bytes:
		return self._suite.encrypt(text.encode())

class Encryption:

	def __init__(self, encryptor: Encryptor) -> None:
		self._encryptor = encryptor

	def encrypt(self, text: str) -> bytes:
		return self._encryptor.encrypt(text)
\end{lstlisting}

\subsection{Visitor}
The visitors purpose is to abstract functionality that can be applied to an aggregate hierarchy of \'Element\' objects. \cite{sourcemaking} This allows the operations to be put into their own class which stops polluting the objects class.
\begin{itemize}
	\item The visitor asks the element to accept it's visit.
	\item The element accepts and passes itself to the visitors visit method.
\end{itemize}

We implemented the Visitor design pattern to add functionality to our requests. We have only implemented json
requests for the time being but would intend to implement XML requests also. The visitor has the functionality
to handle this.

\begin{lstlisting}[language=Python]
class JsonVisitor(ABC):

	@abstractmethod
	@overload
	def visit(self, visitable: JsonRequest) -> str:
		...

class JsonSanitizer(JsonVisitor):

	@overload
	def visit(self, json_request: JsonRequest) -> str:
		return str(json.loads(json_request.text))

class Visitable(ABC):

	@abstractmethod
	def accept(self, visitor: JsonSanitizer) -> None:
		...

class JsonRequest(Visitable):
	
	def accept(self, visitor: JsonSanitizer) -> None:
		self._text = visitor.visit(self)

\end{lstlisting}

\section{Architectural}
Architectural patterns are similar to software design pattern but with a broader scope. The architectural patterns address various issues in software engineering, such as computer hardware performance limitations, high availability and minimization of a business risk. Architectural patterns have often been implemented within software frameworks such as Apache Tomcat and Glassfish.

"\textit{Architecture patterns help define the basic characteristics and behavior of an application. For example, some architecture patterns naturally lend themselves toward highly scalable applications,
whereas other architecture patterns naturally lend themselves toward applications that are highly agile.}" \citep{patterns}

\subsection{Interceptor}
The Interceptor is used for developing frameworks that can be extended transparently, allowing out-of-band services to register with the framework using predefined interfaces. The framework creates  a dispatched and context object. The client creates a concrete interceptor from the interface. The client attaches the interceptor to the dispatcher. When an event occurs in the framework it will pass the context object to the dispatcher which will iterate all registered interceptor, using the observer design pattern method. The context object is passed to the interceptor where the client can then access the internals of the framework.

\subsection{REST}
TODO
